\documentclass{article}


% if you need to pass options to natbib, use, e.g.:
% \PassOptionsToPackage{numbers, compress}{natbib}
% before loading nips_2018

% ready for submission
% \usepackage{nips_2018}

% to compile a preprint version, e.g., for submission to arXiv, add
% add the [preprint] option:
\usepackage[final]{nips_2018}

% to compile a camera-ready version, add the [final] option, e.g.:
% \usepackage[final]{nips_2018}

% to avoid loading the natbib package, add option nonatbib:
% \usepackage[nonatbib]{nips_2018}

\usepackage[utf8]{inputenc} % allow utf-8 input
\usepackage[T1]{fontenc}    % use 8-bit T1 fonts
\usepackage{hyperref}       % hyperlinks
\usepackage{url}            % simple URL typesetting
\usepackage{booktabs}       % professional-quality tables
\usepackage{amsfonts}       % blackboard math symbols
\usepackage{amsmath}
\usepackage{nicefrac}       % compact symbols for 1/2, etc.
\usepackage{microtype}      % microtypography
\usepackage[pdftex]{graphicx}
\graphicspath{ {images/} }



\title{Recurrent Neural Networks for Time Series Prediction}

% The \author macro works with any number of authors. There are two
% commands used to separate the names and addresses of multiple
% authors: \And and \AND.
%
% Using \And between authors leaves it to LaTeX to determine where to
% break the lines. Using \AND forces a line break at that point. So,
% if LaTeX puts 3 of 4 authors names on the first line, and the last
% on the second line, try using \AND instead of \And before the third
% author name.

\author{
  Davide Spallaccini\thanks{\texttt{spallaccini.1642557@studenti.uniroma1.it}} \\
  Department of Computer, Control \\ and Management Engineering\\
  Sapienza University of Rome\\
  Rome, Italy \\
  \And
  Beatrice Bevilacqua\thanks{\texttt{bevilacqua.1645689@studenti.uniroma1.it}} \\
  Department of Computer, Control \\ and Management Engineering\\
  Sapienza University of Rome\\
  Rome, Italy \\
   \\
  %% examples of more authors
  \And
  Anxhelo Xhebraj\thanks{\texttt{xhebraj.1643777@studenti.uniroma1.it}} \\
  Department of Computer, Control \\ and Management Engineering\\
  Sapienza University of Rome\\
  Rome, Italy
  %% Affiliation \\
  %% Address \\
  %% \texttt{email} \\
  %% \AND
  %% Coauthor \\
  %% Affiliation \\
  %% Address \\
  %% \texttt{email} \\
  %% \And
  %% Coauthor \\
  %% Affiliation \\
  %% Address \\
  %% \texttt{email} \\
  %% \And
  %% Coauthor \\
  %% Affiliation \\
  %% Address \\
  %% \texttt{email} \\
}

\begin{document}
% \nipsfinalcopy is no longer used

\maketitle

\begin{abstract}

TODO

\end{abstract}

\section{Introduction and Related Work}
\label{sec:intro}

We will face the problem of time series prediction through Nonlinear autoregressive exogenous models (NARX) whose 
purpose is to predict the current value of a time series based on it's collected knowledge coming from the history of
the time series to predict and the current and past values of a set of exogenous time series. Exogenous time series 
are data that may be correlated to the series under consideration so they are also called "driving series". Of course
a good NARX algorithm should be able to choose which of the series are to be considered "driving" when predicting the
current value. At the same time a good algorithm should be able to capture long-term temporal dependencies.

In this work we analyse a deep-learning-based model that tries to takle with these two issues through the use of 
recurrent neural networks and a carefully combined system of attention mechanisms. We also compare the result with
other possible approaches to the same problem using again recurrent networks and also perform an ablation study that 
wasn't directly reported in the author's paper to validate the effectiveness of the approach proposed.

Models from the past, though effective for some real word applications, do not perform very well since they cannot 
model nonlinear relationships and do not differentiate beween driving terms, or they model nonlinear relationships 
but with a predefined nonlinear relationship which is not necessary the true underlying one.

\section{Dataset}
\label{sec:retrieval}

We used the same dataset as the authors in order to try to compare our results with the ones obtained by them. However
since the paper doesn't mention anything about data normalisation, selection or other kinds of preprocessing we didn't 
get their exact results. We still got results that we could use to compare the different approaches and validate the 
effectiveness of the use of the dual-stage attention mechanism.

The first dataset is SML 2010 ...

The second one is the NASDAQ 100 Stock dataset which is characterised by a large number of driving series ...


\section{Model}
\label{sec:model}

\paragraph{Dual-stage attention model}

The proposed model in our reference paper is composed of an input attention whose output is fed to an encoder RNN. The 
result is the input to a temporal attention and the output of this second attention mechanism is composed with the 
past history of the target time series to the decoder RNN. We notice that there is a fixed-size interval of time from
which we sample a fixed number of timesteps ($T$ henceforward).

The input attention is trained to assign a weight to all the possible input driving series. In particular we have 
that each driving series is represented as a vector of $T$ values, one for each timesteps. For each timestep $t$ the 
input attention assigns a weight to each of the driving series values for that timestep (i.e. at the offset $t$ of the
driving series vector).

The encoder RNN is implemented with a LSTM cell, this choice allows the model to capture long-term dependencies. 
For each timestep we map the result of the first attention to the output $\mathbf{h}_t$ which is the hidden state of 
the LSTM at timestep $t$. The operations involved in this mapping are the ones that take place in the update of a LSTM
cell and can be summarised as follows:


\begin{equation} \label{eq:lstm}
\begin{split}
\mathbf{f}_t &= \sigma (\mathbf{W}_f[\mathbf{h}_{t-1};\mathbf{x}_t] + \mathbf{b}_f) \\
\mathbf{i}_t &= \sigma (\mathbf{W}_i[\mathbf{h}_{t-1};\mathbf{x}_t] + \mathbf{b}_i) \\
\mathbf{o}_t &= \sigma (\mathbf{W}_o[\mathbf{h}_{t-1};\mathbf{x}_t] + \mathbf{b}_o) \\
\mathbf{s}_t &= \mathbf{f}_t \odot \mathbf{s}_t + \mathbf{i}_t 
				\odot \text{tanh}(\mathbf{W}_s[\mathbf{h}_{t-1};\mathbf{x}_t] + \mathbf{b}_s) \\
\mathbf{h}_t &= \mathbf{o}_t \odot \text{tanh}(\mathbf{s}_t)
\end{split}
\end{equation}

Where $[\mathbf{h}_{t-1};\mathbf{x}_t] \in \mathbb{R}^{m + n}$ is a concatenation of the hidden state of the previous 
timestep and the input of the current timestep; $\mathbf{W}_f,\mathbf{W}_i,\mathbf{W}_o,\mathbf{W}_s 
\in \mathbb{R}^{m \times(m+n)}$, and $\mathbf{b}_f, \mathbf{b}_i, \mathbf{b}_o,\mathbf{b}_s \in \mathbb{R}^m$ are 
parameters to learn; $\sigma$ and $\odot$ are a logistic sigmoid function and an element-wise multiplication,
respectively.

The output of the encoder, that is the vectors $\mathbf{h}_t$ are taken as input to the temporal attention.
The temporal attention assigns a weight, for each timestep, to each timestep. Doing so we allow the model to keep
track of old information if it is useful to the prediction. This could not be easy since without attention all this 
information would be packed only into the hidden state of the decoder cell of the previous timestep.

The decoder takes as input a concatenation of weighted sum (different at each timestep) of the hidden states coming 
from the encoder and the value of the target series at timestep $t$. Then, finally, the last hidden state of the decoder
cells concatenated with the last weighted sum of the encoder states is fed to a dense layer that outputs the value 
$y_T$ of the target series for the current timestep.

\paragraph{Implementation details}

We used Tensorflow as the target framework to implement the model just described. The reason for this choice is that we
could easily translate the equations of the model directly into the code and define each step of the model without 
trying to figure out the how to obtain the same results using the high level APIs of Keras. This was because the 
Keras documentation about RNNs to implement this custom attention model was a bit foggy at the beginning so we decided
to step back to Tensorflow. 

For the encoder network we used the \texttt{LSTMCell} class from \texttt{tensorflow.python.ops.rnn\_cell\_impl}.
We initially instantiated a zero state for the cell and the hidden states. Then for each timestep we defined the 
Tensorflow computational graph feeding the \texttt{LSTMCell} with the output of the previous timestep combined with the
output of the attention. During this process we collect the output hidden states of the cell for the various timesteps.
After the proper reshaping this output will be the input of the decoder network.

The decoder network implementation is very similar to the encoder also for the attention part, the key difference where 
the dimensions since the temporal attention works on a different range of inputs. We initially defined weights and bias 
variables for each operation of the attention mechanisms but we later made the code more concise by using the 
\texttt{tensorflow.layers.dense()} API.

In order to set the proper shapes in every part of the implementation of the original model we used a nice facility of 
Tensorflow called Eager Execution that allowed us to see the actual shapes of variables at each step given a sample 
input.

To make the experimentation modular we also defined a data class that loads all the parameters and file paths from 
JSON configuration files.

For the training procedure we use a standard back propagation algorithm since the model is fully differentiable and we 
set the mean squared error as the objective function. We selected the Adam optimizer while for other parameters we 
selected the best performing ones on the validation dataset as described in the following section.

\section{Experiments}

TODO

\section{Ablation Study}

TODO

\section{Conclusion}

TODO

\section*{References}

\small

[1] Yao Qin, Dongjin Song, Haifeng Chen, Wei Cheng, Guofei Jiang, Garrison W. Cottrell. A Dual-Stage Attention-Based Recurrent Neural Network for Time Series Prediction. 2017.

\end{document}
